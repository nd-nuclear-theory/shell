\documentclass[aps,groupedaddress,onecolumn,11pt,floatfix]{revtex4}


% \ifproofpre{twocolumn version}{onecolumn version}
%   for proof (twocolumn) -- #1
%   for preprint -- #2
\newcommand{\ifproofpre}[2]{#1}

% float placement 
\setcounter{topnumber}{99}
\setcounter{bottomnumber}{99}
\setcounter{totalnumber}{99}
\renewcommand{\topfraction}{.99}
\renewcommand{\bottomfraction}{0.99}
\renewcommand{\textfraction}{0.01}
\setcounter{dbltopnumber}{99}
\renewcommand{\dbltopfraction}{.99}
%%\renewcommand{\floatpagefraction}{0.01}
%%\renewcommand{\dblfloatpagefraction}{0.01}

% standard packages
\usepackage{graphicx}
\usepackage{amsmath}
\usepackage{amssymb}
\usepackage{mathtools}
\usepackage{isotope}
%%\usepackage{dsfont} % for \mathds{1}
\newcommand{\mathds}{}

% mc packages
\usepackage{wigner}
\usepackage{liealg}

% notation

\newcommand{\MeV}{{\mathrm{MeV}}}
\newcommand{\keV}{{\mathrm{keV}}}
\newcommand{\fm}{{\mathrm{fm}}}

\newcommand{\Nmax}{N_\text{max}}
\newcommand{\Trel}{T_\text{rel}}
\newcommand{\Arel}{A_\text{rel}}

%% \newcommand{\Nrel}{{N_\text{rel}}}
%% \newcommand{\Ncm}{{N_\text{c.m.}}}
%% \newcommand{\nrel}{{n_\text{rel}}}
%% \newcommand{\ncm}{{n_\text{c.m.}}}
%% \newcommand{\lrel}{{l_\text{rel}}}
%% \newcommand{\lcm}{{l_\text{c.m.}}}

% abbreviated
\newcommand{\Nr}{N_r}
\newcommand{\Nc}{N_c}
\newcommand{\nr}{n_r}
\newcommand{\nc}{n_c}
\newcommand{\lr}{l_r}
\newcommand{\lc}{l_c}
\newcommand{\Jr}{J_r}


\newcommand{\unity}{\mathds{1}}
\newcommand{\unitycm}{{\mathds{1}_{\text{c.m.}}}}



\newcommand{\hw}{{\hbar\omega}}
\newcommand{\bint}{{b_{\text{int}}}}
\newcommand{\hwint}{{\hbar\omega_{\text{int}}}}
\newcommand{\Ncut}{{N_\text{cut}}}  
\newcommand{\Ncm}{{N_\text{c.m.}}}  
\newcommand{\Nex}{{N_\text{ex}}}  
\newcommand{\Nv}{{N_v}}  

\newcommand{\Einf}{{E_\infty}}  
\newcommand{\kinf}{{k_\infty}}  
\newcommand{\rinf}{r_\infty}  % used in square, so do not nest braces

\newcommand{\scrN}{{\mathcal{N}}}  % lowercase missing

%% \newcommand{\rp}{r_p}  
%% \newcommand{\rn}{r_n}  
%% \newcommand{\rm}{r_m}    % symbol clash with \rm font directive

% references
\newcommand{\pref}[1]{(\ref{#1})}

% recoupling

% \overlapZ{j1}{j2}{j3}{J12}{J13}{J}
%%\overlap[3]{\begin{matrix}j_1&&j_2&&j_3\\&\mathclap{J_{12}}\\&&J\end{matrix}}{\begin{matrix}j_1&&j_2&&j_3\\&&\mathclap{J_{13}}\\&&J\end{matrix}}
\newcommand{\overlapZ}[6]{\overlap[3]{\begin{matrix}#1&&#2&&#3\\&\mathclap{#4}\\&&#6\end{matrix}}{\begin{matrix}#1&&#2&&#3\\&&\mathclap{#5}\\&&#6\end{matrix}}}

% \ketcoupled{j1}{j2}{j3}{J12}{J}
\newcommand{\ketcoupled}[5]{\ket[3]{\begin{matrix}#1&&#2&&#3\\&\mathclap{#4}\\&&#5\end{matrix}}}

% \ketcoupledmid{j1}{j2}{j3}{J13}{J}
\newcommand{\ketcoupledmid}[5]{\ket[3]{\begin{matrix}#1&&#2&&#3\\&&\mathclap{#4}\\&&#5\end{matrix}}}


\begin{document}

\title{\texttt{moshinsky}: Moshinsky transformation}

\author{Mark A. Caprio}
\affiliation{Department of Physics, University of Notre Dame, Notre Dame, Indiana 46556-5670, USA}

\date{\today}

\begin{abstract}
\textit{Ab initio} calculations of nuclei face the challenge of
simultaneously describing strong short-range internucleon correlations
and the long-range properties of weakly-bound halo nucleons.  Natural
orbitals, which diagonalize the one-body density matrix, provide a
basis which is better matched to the physical structure of the
many-body wave function.  We demonstrate that the use of natural
orbitals significantly improves convergence for \textit{ab initio}
no-core configuration interaction calculations of the neutron halo
nucleus $\isotope[6]{He}$, relative to the traditional oscillator
basis.
\end{abstract}


\maketitle


%%\section{Introduction}
%%\label{sec-intro}

\section{Moshinsky transformation}

\subsection{Transformation of $LSJT$-scheme matrix elements}

We consider a relative operator $\Arel^{J_0T_0g_0}$ of definite
angular momentum $J_0$, isospin $T_0$ and parity $P_0$, which we will
find it convenient to express
in terms of a grade label $g_0$ as $P_0=(-)^{g_0}$.  The
corresponding operator $A^{J_0T_0P_0}$ on the full two-body fermionic
space, which 
 has these same symmetry
properties, is simply $A\equiv\Arel\times\unitycm$.

The Moshinsky brackets $\toverlap{\Nr\lr\Nc\lc;L}{N_1l_1N_2l_2;L}$ are
the unitary transformation brackets relating relative-c.m. states and two-body
states.  In particular, since we are working with fermions, it is
important to realize that the Moshinsky brackets are defined for
\textit{distinguishable} particle states, and we will need to
explicitly account for antisymmetry later.  We follow Negele's
convention of using a parenthesis to denote distinguishable particle
states here, as $\tpket{N_1 l_1 N_2 l_2;LSJTg}$.  Thus, in particular,
\begin{equation}
\label{eqn-moshinsky-xform-ket-dist}
\tpket{N_1 l_1 N_2 l_2;LSJTg}
=\sum_{\substack{\Nr\lr\\\Nc\lc}}
\toverlap{\Nr\lr\Nc\lc;L}{N_1l_1N_2l_2;L}
\,
\tket{\Nr \lr \Nc \lc;LSJTg}.
\end{equation}
There are a few things worth noting in this relation:

(1) The non-orbital angular momentum labels $SJT$ are purely spectator
labels in the summation.

(2) The values for the $l$ labels follow from those for the $N$ labels
by the usual oscillator branching rule [\textit{e.g.}, for the
  relative oscillator labels, $\lr= (\Nr \bmod 2), (\Nr
  \bmod 2)+2,\ldots,\Nr$].  Then the
Moshinsky brackets enforce triangularity constraints $\Delta(\lr\lc L)$
and $\Delta(l_1l_2L)$.

(3) The Moshinsky bracket
preserves total
oscillator quanta
\begin{equation}
\Nr+\Nc = N_1 + N_2 \quad(\equiv N).
\end{equation}
Therefore, in the summation, the values of the indices $\Nr$ and $\Nc$
are in direct correspondence.  That is, if we sum over $\Nr$, then the
summation over $\Nc$ is trivial, and \textit{vice versa}.

(4) The parity of a two-body or relative-c.m.~state is determined by
the total oscillator quantum number.  Consequently, the Moshinsky
bracket also conserves parity.  Again using a grade label $g$ to indicate the parity
[$P=(-)^g$], we have, in terms of relative-c.m.~or two-body labels, respectively,
\begin{equation}
\begin{aligned}
g&\cong \lr + \lc &&\cong \Nr+\Nc &&\cong N\\
g&\cong l_1 + l_2 &&\cong N_1+N_2 &&\cong N,
\end{aligned}
\end{equation}
where congruence indicates equivalence modulo~$2$.  The parity label
is thus redundant to the oscillator labels.  However, we include the
parity labels explicitly on the states in this derivation, since the
classification of states according to their parity subspaces is an
essential aspect of the basis indexing in the implementation of the
Moshinsky transformation.

(5) Although the 
Moshinsky bracket is traditionally notated in terms of the
\textit{radial} quantum numbers ($n_1$, $n_2$, $\nr$, and $\nc$), we have
chosen to notate it in terms of the \textit{principal} oscillator
quantum numbers ($N_1$, $N_2$, $\Nr$, and $\Nc$), to better reflect the above
conservation properties.


We express two-body matrix
elements (TBMEs) in terms of relative-c.m.~matrix elements by
transforming the bra and ket spaces separately.
In particular, we relate $JT$-reduced
matrix elements (RMEs) between $LSJT$-scheme states:
\begin{multline}
\label{eqn-moshinsky-xform-op-dist-redundant}
\tprme{N_1' l_1' N_2' l_2';L'S'J'T'g'}{A^{J_0T_0g_0}}{N_1 l_1 N_2 l_2;LSJTg}
\\
=\sum_{\substack{\Nr'\lr'[\Nc'\lc']\\ \Nr\lr\Nc\lc}}
\toverlap{\Nr'\lr'\Nc'\lc';L'}{N_1'l_1'N_2'l_2';L'}
\toverlap{\Nr\lr\Nc\lc;L}{N_1l_1N_2l_2;L}
\\
\times
\trme{\Nr' \Nr' \Nc' \lc';L'S'J'T'g'}{A^{J_0T_0g_0}}{\Nr \lr \Nc \lc; LSJTg}.
\end{multline}
The symmetry properties of the operator enforce $\Delta{J'J_0J}$,
$\Delta{T'T_oT}$, and $g'+g_0+g\cong0$.
 Again, it will be important to realize that, so far,
we have been working with distinguishable-particle states: on the LHS, note that we have
\textit{distinguishable}-particle TBMEs, and, more subtly, we have not
imposed any restrictions on the summation due to antisymmetry
constraints on the labels of relative states.

The particular
Wigner-Eckhart normalization convention adopted for the $JT$-reduced
matrix elements should
be kept in mind, since it enters into the derivation at various points
below, both on the angular momentum side and the isospin side.  Under
the \textit{group theoretic} convention (used by Rose),
\begin{equation}
\label{eqn-we-group}
\me[2]{J'T'\\M_J'M_T'}{A^{J_0T_0}_{M_{J0}M_{T0}}}{JT\\M_JM_T}
=
\cg{J}{M_J}{J_0}{M_{J0}}{J'}{M_J'}
\cg{T}{M_T}{T_0}{M_{T0}}{T'}{M_T'}
\trme{J'T'}{A^{J_0T_0}}{JT},
\end{equation}
 while, under the common \textit{angular momentum} convention of
 Racah (used by Edmonds),
\begin{equation}
\label{eqn-we-racah}
\me[2]{J'T'\\M_J'M_T'}{A^{J_0T_0}_{M_{J0}M_{T0}}}{JT\\M_JM_T}
=
(-)^{2J_0+2T_0} \frac1{\hat{J}'\hat{T}'}
\cg{J}{M_J}{J_0}{M_{J0}}{J'}{M_J'}
\cg{T}{M_T}{T_0}{M_{T0}}{T'}{M_T'}
\trme{J'T'}{A^{J_0T_0}}{JT}.
\end{equation}
Thus, reduced matrix elements under angular momentum (a.m.) and group
theory (g.t.) conventions are related by
\begin{equation}
\trme{J'T'}{A^{J_0T_0}}{JT}_{\text{a.m.}}=
(-)^{2J_0+2T_0} {\hat{J}'\hat{T}'}
\trme{J'T'}{A^{J_0T_0}}{JT}_{\text{g.t.}}.
\end{equation}
For the identity operator, 
$\trme{J'T'}{\unity^{00}}{JT}_{\text{g.t.}}= 1$, but
$\trme{J'T'}{\unity^{00}}{JT}_{\text{a.m.}}= {\hat{J}'\hat{T}'}$.

We shall adhere to the group theoretic convention
of~(\ref{eqn-we-group}) below, but beware that
Racah's conventions are widely followed in the shell model literature
(\textit{e.g.}, Talmi and Suhonen).  
The phase factors
in~(\ref{eqn-we-racah}), which arise
from the relation between $3j$ symbols and Clebsch-Gordan
coefficients, are only relevant to operators carrying an odd fermion
number and thus do not affect the two-body operators under
consideration here in the Moshinsky transformation.  However, the dimension factors
[$\hat{J}\equiv(2J+1)^{1/2}$] do enter into the derivation below: in
Racah's reduction formula for matrix elements of products of operators
and in the branching from isospin scheme to $pn$ scheme.  


Since $A$ is obtained from a \textit{relative} operator
$\Arel$, the relative-c.m.~matrix element imposes the Kronecker delta
constraint $(\Nc'\lc')=(\Nc\lc)$.  It may therefore seem natural to eliminate
the summation over $(\Nc'\lc')$ in~\ref{eqn-moshinsky-xform-op-dist-redundant}, to obtain
\begin{multline}
\label{eqn-moshinsky-xform-op-dist-NcmLcm}
\tprme{N_1' l_1' N_2' l_2';L'S'J'T'g'}{A^{J_0T_0g_0}}{N_1 l_1 N_2
  l_2;LSJTg}
\\
=\sum_{\Nc\lc}\sum_{\substack{\Nr'\lr'\\ \Nr\lr}}
\toverlap{\Nr'\lr'\Nc\lc;L'}{N_1'l_1'N_2'l_2';L'}
\toverlap{\Nr\lr\Nc\lc;L}{N_1l_1N_2l_2;L}
\\
\times
\trme{\Nr' \Nr' \Nc \lc;L'S'J'T'g'}{A^{J_0T_0g_0}}{\Nr \lr \Nc \lc; LSJTg}.
\end{multline}

However, we shall find it useful to base our implementation on the
original expression~\pref{eqn-moshinsky-xform-op-dist-NcmLcm} for the
transformation, which reflects the simple ``matrix-matrix-matrix
multiplication'' structure which one expects of a similarity
transformation.  The constraint that the Moshinsky brackets conserve
$N$ enforces a block-diagonal structure to the transformation
matrices, if we organize our bases by $N$.  While the reduction in
the number of summation indices obtained
in~\pref{eqn-moshinsky-xform-op-dist-NcmLcm} may seem like a
``natural'' simplification, it has the disadvantage that it destroys
the simple matrix-matrix-matrix mutliplication structure.  The RHS
of~\pref{eqn-moshinsky-xform-op-dist-NcmLcm} may still be interpreted
as a \textit{sum}, over $(\Nc\lc)$, of matrix-matrix-matrix products,
where the transformation matrices are indexed by
$(\Nr\lr,N_1l_1N_2l_2)$ at fixed $(\Nc\lc)$, and the untransformed
matrices are indexed by the relative quantum numbers only, as
$(\Nr'\lr',\Nr\lr)$, at fixed $(\Nc\lc)$.  However, this formulation
is not as conducive to making use of the block structure in $N$.

We must still evaluate the relative-c.m.~matrix elements $\trme{\Nr'
  \lr' \Nc \lc;L'S'J'T'g'}{A^{J_0T_0g_0}}{\Nr \lr \Nc \lc; LSJTg}$ in
terms of the underlying relative matrix elements $\trme{\Nr' \Nr'
  ;\lr' S'\Jr'T'g'}{\Arel^{J_0T_0g_0}}{\Nr \lr; \lr S\Jr Tg}$.  These
relative matrix elements are the starting point for the Moshinsky
transformation.  The difference between relative and
relative-c.m.~matrix elements arises from the presence ``spectator''
c.m.~excitations.  Although the operator $A$ acts as the identity on
the c.m.~space (recall $A\equiv\Arel\times\unitycm$), the angular
momentum of the c.m.~excitation affects the value of the
relative-c.m.~matrix element through the angular momentum coupling.

The basic idea is to use
Racah's reduction formula for reduced matrix elements of the product
of two spherical tensor operators acting on two different spaces, to
factorize $\trme{}{A^{J_0}}{}\propto\trme{}{\Arel^{J_0}}{}\trme{}{\unity^{0}}{}$.
If the second operator is the
identity operator, Racah's reduction formula becomes, under the group theoretic convention, 
\begin{equation}
\rme[2]{J_1' J_2\\J'}{(A_1^{\lambda}\times\unity^0)^\lambda}{J_1J_2\\J}
=(-)^{J_1'+J_2+J+\lambda}\hat{J}_1'\hat{J}\sixj{J_1'}{J'}{J_2}{J}{J_1}{\lambda}
\trme{J_1'}{A_1^\lambda}{J_1}.
\end{equation}
We shall denote this prefactor for first-system operators by
\begin{equation}
R_{1,\text{g.t.}}(J_1'J';J_1J;J_2\lambda)\equiv(-)^{J_1'+J_2+J+\lambda}\hat{J}_1'\hat{J}\sixj{J_1'}{J'}{J_2}{J}{J_1}{\lambda}.
\end{equation}
Alternatively, under the angular momentum convention,
\begin{equation}
\rme[2]{J_1' J_2\\J'}{(A_1^{\lambda}\times\unity^0)^\lambda}{J_1J_2\\J}
=(-)^{J_1'+J_2+J+\lambda}\hat{J}'\hat{J}\sixj{J_1'}{J'}{J_2}{J}{J_1}{\lambda}
\trme{J_1'}{A_1^\lambda}{J_1}.
\end{equation}

However, to make use of such a factorization in evaluating the
relative-c.m.~matrix element, we must first separate out the angular
momentum of the c.m.~excitation from the other angular momenta in the
coupling, so that it can fulfil the role of the ``second system''
angular momentum.  
In general, the
$(12)3-(13)2$ recoupling transformation bracket is given by
\begin{equation}
\overlapZ{j_1}{j_2}{j_3}{J_{12}}{J_{13}}{J}
=(-)^{j_2+j_3+J_{12}+J_{13}}
\hat{J}_{12}\hat{J}_{13}\sixj{j_1}{j_2}{J_{12}}{J}{j_3}{J_{13}}
\equiv Z(j_1j_2j_3;J_{12}J_{13};J),
\end{equation}
where the horizontal alignment of the angular momentum labels on the
LHS is intended to indicate that it is $j_1$ and $j_2$ which couple to give $J_{12}$ in
the bra, but that it is $j_1$ and $j_3$ which couple to give $J_{13}$ in the ket.  We
adopt the letter $Z$ for this $(12)3-(13)2$ ``unitary recoupling
coefficient'', following Millener's usage in the $\grpsu{3}$ scheme,
although in Millener's notation the arrangement of arguments would be
$Z(j_2j_1Jj_3J_{12}J_{13})$, which is somewhat less transparent for
present purposes.  
In particular, for the present problem the
angular momenta recouple as
\begin{equation}
\begin{aligned}
\ketcoupled{\lr}{\lc}{S}{L}{J}
&=\sum_{\Jr} \overlapZ{\lr}{\lc}{S}{L}{\Jr}{J}\,\ketcoupledmid{\lr}{\lc}{S}{\Jr}{J}\\
&=\sum_{\Jr}Z(\lr\lc S;L\Jr;J)\,\ketcoupled{\lr}{S}{\lc}{\Jr}{J}
.
\end{aligned}
\end{equation}
We thus have
\begin{multline}
\trme{\Nr' \lr' \Nc \lc;L'S'J'T'g'}{A^{J_0T_0g_0}}{\Nr \lr \Nc \lc; LSJTg}
\\=\sum_{\Jr'\Jr}
Z(\lr'\lc S';L'\Jr';J')
Z(\lr\lc S;L\Jr;J)
R_{1,\text{g.t.}}(\Jr' J';\Jr J;\lc J_0)
\\
\times
\trme{\Nr' \Nr' ;\lr' S'\Jr'T'g'}{\Arel^{J_0T_0g_0}}{\Nr \lr; \lr S\Jr Tg}.
\end{multline}

\subsection{Recoupling to $jjJT$-scheme matrix elements}

\subsection{Accounting for the fermionic nature of the states}

\subsection{Branching to $jjJpn$-scheme matrix elements}

\subsection{Summary of basis indexing schemes}

%%\section*{Acknowledgements}
%%\label{sec-acknowledgements} 

\bibliographystyle{apsrev4-1}
%%\bibliographystyle{apsrevm}
\bibliography{master,mc,theory,expt,data,books,proc,misc,natorb}




%%%%%%%%%%%%%%%%%%%%%%%%%%%%%%%%%%%%%%%%%%%%%%%%%%%%%%%%%%%%%%%%%%%%%%%%%%%%%%%%%%%%%%%%%%%%%%%%%%%%%%%%%%

\end{document} 
